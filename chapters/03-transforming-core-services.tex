\chapter{Transforming Your Core Services}\label{ch:transforming-your-core-services}

%! suppress = LineBreak
\begin{warningblock}
    This is an Early Release. You're getting the raw and unedited content as I write. I'm doing this, so you can take advantage of the content
    before the official release, AND you can share critical feedback (plus, I include you in the credits of the official release)
    To get notified when I add new section(s), \href{https://discord.gg/X2USgYTB}{join the Business Automators discord community}
\end{warningblock}
\begin{importantblock}
    If you found a problem, \href{https://discord.gg/X2USgYTB}{drop a comment in the discord community} or  \href{mailto:dele@protomated.com}{email dele@protomated.com}.
\end{importantblock}


%
%

As an IT consultant, your ability to efficiently gather requirements, prototype solutions, and present data can set you apart from the competition.\ In this chapter, we'll explore how to leverage no-code tools to revolutionize these core services, making your consulting practice more efficient and effective.


\section{Automating Requirements Gathering}

Let's dive deep into each step of our automated requirements gathering workflow, providing detailed instructions on how to set up each part using n8n, NoCoDB, and BudiBase.

\subsection{Step 1: Initiate Project and Define Business Goals}

In this step, we'll create a workflow that automates the project initiation process.

1. Create a Google Form for project initiation:
\begin{itemize}
    \item Include fields for project name, description, key objectives, and expected outcomes
    \item Add a section for defining business goals
\end{itemize}

[PLACEHOLDER: Screenshot of the Google Form for project initiation]

2. Set up an n8n workflow:
\begin{itemize}
    \item Start with a "Google Forms Trigger" node
    \item Add a "NoCoDB" node to create a new project record
\end{itemize}

[PLACEHOLDER: Screenshot of n8n workflow showing Google Forms Trigger and NoCoDB nodes]

3. Configure the NoCoDB node:
\begin{itemize}
    \item Create a "Projects" table in NoCoDB with fields matching your Google Form
    \item In n8n, map the Google Form responses to the corresponding NoCoDB fields
\end{itemize}

[PLACEHOLDER: Screenshot of NoCoDB node configuration in n8n]

4. Add a "Send Email" node to notify relevant team members about the new project

[PLACEHOLDER: Screenshot of completed n8n workflow for project initiation]

\subsection{Step 2: Stakeholder Identification and Mapping}

Now, let's automate the process of identifying stakeholders and mapping them to business goals.

1. Create a Google Form for stakeholder identification:
\begin{itemize}
    \item Include fields for name, role, department, contact information, and influence level
    \item Add a multi-select field for associated business goals
\end{itemize}

[PLACEHOLDER: Screenshot of Google Form for stakeholder identification]

2. Set up an n8n workflow:
\begin{itemize}
    \item Start with a "Google Forms Trigger" node
    \item Add a "NoCoDB" node to store stakeholder information
    \item Include a "Function" node to categorize stakeholders based on their responses
\end{itemize}

[PLACEHOLDER: Screenshot of n8n workflow for stakeholder identification]

3. Configure the Function node to categorize stakeholders:
\begin{lstlisting}[language=JavaScript]
    const influenceLevel = $input.body['Influence Level'];
    const goals = $input.body['Associated Goals'];

   let category = '';
   if (influenceLevel === 'High' && goals.length > 2) {
     category = 'Key Player';
   } else if (influenceLevel === 'High') {
     category = 'Meet Their Needs';
   } else if (goals.length > 2) {
     category = 'Keep Informed';
   } else {
     category = 'Monitor';
   }

   return {
     category: category,
     ...input.body
   };
\end{lstlisting}

[PLACEHOLDER: Screenshot of Function node configuration]

4. Add a "Chart" node to create a visual stakeholder map:
\begin{itemize}
    \item Use a scatter plot with influence level on one axis and number of associated goals on the other
    \item Color-code points based on the stakeholder category
\end{itemize}

[PLACEHOLDER: Screenshot of Chart node configuration and resulting stakeholder map]

\subsection{Step 3: Requirements Elicitation}

Let's implement the Triplet Questioning technique through automation.

1. Create a series of Google Forms for Triplet Questioning:
\begin{itemize}
    \item Form 1: "What is your requirement?"
    \item Form 2: "What does that give you of value?"
    \item Form 3: "Which value is most important?"
\end{itemize}

[PLACEHOLDER: Screenshots of the three Google Forms]

2. Set up an n8n workflow:
\begin{itemize}
    \item Use three "Google Forms Trigger" nodes, one for each form
    \item Add a "NoCoDB" node to store responses
    \item Include a "Send Email" node to trigger the next question in the sequence
\end{itemize}

[PLACEHOLDER: Screenshot of n8n workflow for Triplet Questioning]

3. Configure the email nodes to include links to the next form in the sequence

[PLACEHOLDER: Screenshot of email node configuration]
%
4. Add a "Function" node to generate follow-up questions based on responses:
\begin{lstlisting}[language=Javascript]
    const requirement = $input.body['Requirement'];
    const value = $input.body['Value'];

    let followUpQuestion = '';
    if (value.toLowerCase()
                .includes('efficiency')) {
    followUpQuestion = `How would improving efficiency in "${requirement}" impact your daily operations?`;
    } else if (value.toLowerCase().includes('cost')) {
    followUpQuestion = `Can you quantify the potential cost savings related to "${requirement}"?`;
    } else {
    followUpQuestion = `Could you elaborate on how "${value}" ties into your overall business objectives?`;
    }

    return { followUpQuestion };
\end{lstlisting}

[PLACEHOLDER: Screenshot of Function node for generating follow-up questions]

\subsection{Step 4: Requirements Documentation}

Now, let's automate the creation of requirement records.

1. Create a Google Docs template for requirements documentation:
\begin{itemize}
    \item Include sections for requirement description, associated value, priority, and stakeholders
    \item Add placeholders for dynamic content (e.g., \{\{REQUIREMENT\}\}, \{\{VALUE\}\}, etc.)
\end{itemize}

[PLACEHOLDER: Screenshot of Google Docs template]
%
2. Set up an n8n workflow:
\begin{itemize}
    \item Start with a "NoCoDB Trigger" node to detect new requirements
    \item Add a "Google Docs" node to create a new document from the template
    \item Include a "Function" node to generate a unique identifier for each requirement
\end{itemize}

[PLACEHOLDER: Screenshot of n8n workflow for requirements documentation]

3. Configure the Google Docs node:
\begin{itemize}
    \item Map NoCoDB fields to the placeholders in your template
    \item Set the document name to include the unique identifier
\end{itemize}

[PLACEHOLDER: Screenshot of Google Docs node configuration]

4. Add a "NoCoDB" node to update the requirement record with the document link

[PLACEHOLDER: Screenshot of NoCoDB node for updating requirement record]

\subsection{Step 5: Requirements Validation}

Let's streamline the validation process with stakeholders.

1. Create a Google Form for requirement feedback:
\begin{itemize}
    \item Include fields for requirement ID, clarity rating, completeness rating, and comments
\end{itemize}

[PLACEHOLDER: Screenshot of requirement feedback form]

2. Set up an n8n workflow:
\begin{itemize}
    \item Start with a "Schedule Trigger" node to run daily
    \item Add a "NoCoDB" node to fetch requirements needing validation
    \item Include a "Loop" node to process each requirement
    \item Add a "Send Email" node within the loop to send validation requests
\end{itemize}

[PLACEHOLDER: Screenshot of n8n workflow for requirement validation]

3. Configure the email node:
\begin{itemize}
    \item Include the requirement details and document link in the email body
    \item Add a link to the feedback form
\end{itemize}

[PLACEHOLDER: Screenshot of email node configuration]

4. Add a "Google Forms Trigger" node to process feedback:
\begin{itemize}
    \item Use a "NoCoDB" node to update the requirement record with feedback
    \item Include a "Function" node to determine if further revision is needed based on feedback scores
\end{itemize}

[PLACEHOLDER: Screenshot of feedback processing workflow]

\subsection{Step 6: Requirements Management}

Now, let's create a real-time dashboard for requirements management using BudiBase.

1. Connect BudiBase to your NoCoDB database:
\begin{itemize}
    \item Set up a new data source in BudiBase pointing to your NoCoDB instance
    \item Import the "Requirements" table
\end{itemize}

[PLACEHOLDER: Screenshot of BudiBase data source configuration]

2. Create a new BudiBase app:
\begin{itemize}
    \item Start with a blank template
    \item Add a table component to display all requirements
    \item Include filter options for status, priority, and stakeholder
\end{itemize}

[PLACEHOLDER: Screenshot of BudiBase app design interface]

3. Add visualizations:
\begin{itemize}
    \item Create a pie chart showing requirements by status
    \item Add a bar chart displaying requirements by priority
    \item Include a line chart showing requirements added over time
\end{itemize}

[PLACEHOLDER: Screenshot of BudiBase dashboard with visualizations]

4. Set up n8n to keep the dashboard updated:
\begin{itemize}
    \item Create a workflow with a "Schedule Trigger" node to run hourly
    \item Add a "NoCoDB" node to fetch updated requirement data
    \item Include a "BudiBase" node to update the app's data
\end{itemize}

[PLACEHOLDER: Screenshot of n8n workflow for updating BudiBase dashboard]

\subsection{Step 7: Centralized Governance}

Finally, let's ensure oversight and consistency through automated reporting and centralized documentation.

1. Set up a Google Drive folder structure for project documentation:
\begin{itemize}
    \item Create folders for each project
    \item Include subfolders for requirements, stakeholder information, and reports
\end{itemize}

[PLACEHOLDER: Screenshot of Google Drive folder structure]

2. Create an n8n workflow for automated reporting:
\begin{itemize}
    \item Use a "Schedule Trigger" node to run weekly
    \item Add "NoCoDB" nodes to fetch project and requirement data
    \item Include a "Google Sheets" node to create a summary report
    \item Add a "Send Email" node to distribute the report to the steering committee
\end{itemize}

[PLACEHOLDER: Screenshot of n8n workflow for automated reporting]

3. Configure the Google Sheets node:
\begin{itemize}
    \item Create a template for the weekly report
    \item Map fetched data to appropriate cells in the spreadsheet
\end{itemize}

[PLACEHOLDER: Screenshot of Google Sheets node configuration]

4. Set up document organization workflow:
\begin{itemize}
    \item Create an n8n workflow triggered by new document creation
    \item Use a "Switch" node to determine the document type
    \item Add "Move File" nodes to place documents in the correct Google Drive folders
\end{itemize}

[PLACEHOLDER: Screenshot of document organization workflow]

By implementing this comprehensive, automated requirements gathering system, you'll significantly streamline your process, reduce errors, and impress clients with your efficiency and organization. Remember to test each component thoroughly and gather feedback from your team to continually refine and improve the system.


\section{Rapid Prototyping Techniques That Wow Clients}

Once you've gathered requirements, the next step is creating a prototype to validate ideas and get client feedback. No-code tools excel at rapid prototyping, allowing you to create impressive demos quickly. Let's walk through the process of building a functional prototype using BudiBase and enhancing it with n8n workflows.

\subsection{Using BudiBase for Quick UI Prototypes}

BudiBase is an excellent tool for creating functional UI prototypes quickly. We'll create a simple client management dashboard as an example.

\subsubsection{Step 1: Set Up Your BudiBase Environment}

1. If you haven't already, sign up for a BudiBase account at https://budibase.com/
2. Once logged in, click on "Create new app"
3. Choose "Start from scratch"
4. Name your app "Client Management Dashboard" and click "Create app"

[PLACEHOLDER: Screenshot of BudiBase "Create new app" screen]

\subsubsection{Step 2: Create a Data Source}

1. In your new app, go to the "Data" section in the left sidebar
2. Click "Add new data source"
3. For this example, choose "CSV"
4. Upload a sample CSV file with client data (columns: Name, Email, Company, Last Contact Date, Status)
5. Click "Import data"

[PLACEHOLDER: Screenshot of BudiBase data source creation screen]

\subsubsection{Step 3: Build the Main Dashboard}

1. Go to the "Design" section in the left sidebar
2. Click "Add screen" and choose "Blank screen"
3. Name it "Dashboard" and click "Create screen"
4. From the components panel on the right, drag a "Container" onto your blank screen
5. Inside the container, add a "Headline" component and set the text to "Client Management Dashboard"

[PLACEHOLDER: Screenshot of BudiBase design screen with initial dashboard layout]

\subsubsection{Step 4: Add a Client List}

1. Drag a "Table" component into your container, below the headline
2. In the component settings on the right, set the data source to your imported CSV
3. Choose the columns you want to display (e.g., Name, Company, Status)
4. Add a "Button" component to the table and label it "View Details"

[PLACEHOLDER: Screenshot of BudiBase screen with table component added]

\subsubsection{Step 5: Create a Client Details Screen}

1. Add another blank screen and name it "Client Details"
2. Add a "Container" component
3. Inside the container, add "Text" components for each piece of client information (Name, Email, Company, etc.)
4. Bind these text components to the respective data fields

[PLACEHOLDER: Screenshot of Client Details screen layout]

\subsubsection{Step 6: Add Navigation}

1. Return to the Dashboard screen
2. Select the "View Details" button in the table
3. In the settings panel, under "Actions", choose "Navigate to screen" and select the Client Details screen
4. Set up a parameter to pass the selected client's ID to the details screen

[PLACEHOLDER: Screenshot of button action configuration]

\subsubsection{Step 7: Enhance with Charts}

1. On the Dashboard screen, add a new container below the client list
2. Drag a "Chart" component into this container
3. In the chart settings, choose "Pie Chart" and set the data source to your client CSV
4. Configure the chart to show the distribution of clients by status

[PLACEHOLDER: Screenshot of dashboard with added chart]

\subsection{Creating Interactive Workflows with n8n}

Now let's enhance our prototype with some automated workflows using n8n.

\subsubsection{Step 1: Set Up n8n}

1. If you haven't already, install n8n locally or sign up for n8n.cloud
2. Open n8n and click "Create new workflow"
3. Name your workflow "Client Update Automation"

[PLACEHOLDER: Screenshot of n8n new workflow creation]

\subsubsection{Step 2: Create a Trigger Node}

1. In the node panel, search for "Webhook" and add it to your workflow
2. Configure the webhook to receive POST requests
3. Save the generated webhook URL - we'll use this in BudiBase

[PLACEHOLDER: Screenshot of Webhook node configuration]

\subsubsection{Step 3: Add Processing Nodes}

1. Add a "Function" node after the Webhook node
2. In the Function node, add code to format the incoming data:

\begin{lstlisting}
return {
  json: {
    clientName: $input.body.clientName,
    newStatus: $input.body.newStatus,
    updateDate: new Date().toISOString()
  }
};
\end{lstlisting}

[PLACEHOLDER: Screenshot of Function node configuration]

\subsubsection{Step 4: Add an Action Node}

1. Add a "Send Email" node (you may need to set up an email service integration)
2. Configure the email node to send an update to a specified address
3. Use the data from the Function node to populate the email content

[PLACEHOLDER: Screenshot of Send Email node configuration]

\subsubsection{Step 5: Integrate n8n with BudiBase}

1. Return to your BudiBase app
2. On the Client Details screen, add an "Update Status" button
3. In the button's action settings, choose "Fetch data"
4. Set the URL to your n8n webhook URL
5. Configure the request to send the client's name and new status

[PLACEHOLDER: Screenshot of BudiBase button configuration for n8n integration]

\subsection{Testing Your Prototype}

1. In BudiBase, use the "Preview" feature to test your app
2. Navigate through the dashboard, view client details, and try updating a client's status
3. Check that the n8n workflow is triggered and an email is sent

[PLACEHOLDER: Screenshot of BudiBase app preview]

\subsection{Prototype Presentation Best Practices}

When presenting your prototype to clients:

1. \textbf{Set the context:} Explain that this is a rapid prototype to visualize concepts, not a final product.

2. \textbf{Focus on functionality:} Emphasize how the prototype addresses their specific requirements.

3. \textbf{Encourage interaction:} Let clients click through the prototype themselves.

4. \textbf{Highlight flexibility:} Demonstrate how easily elements can be adjusted based on feedback.

5. \textbf{Discuss next steps:} Use the prototype as a basis for refining requirements and planning development.

By following these steps, you can quickly create an impressive, functional prototype that will wow your clients and provide a solid foundation for further development. Remember, the key is to iterate rapidly based on feedback, continuously refining the prototype to meet your client's needs.

\section{Conclusion}

By leveraging no-code tools to automate requirements gathering, streamline prototyping, and create dynamic dashboards, you can transform your core IT consulting services. These techniques not only save you time but also impress clients with your efficiency and professionalism.

In the next chapter, we'll explore how to scale your practice using these automated solutions, allowing you to take on more clients without proportionally increasing your workload.
%
\textbf{Action Item}: Take one of your current projects and implement the automated requirements gathering workflow we discussed. Note how it impacts your efficiency and client satisfaction.

